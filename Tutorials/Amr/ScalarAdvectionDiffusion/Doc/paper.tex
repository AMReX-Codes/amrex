\documentclass{article}

\bibliographystyle{plain}

\usepackage{epsfig}
\usepackage{amssymb}
\usepackage{amsmath}

\newcommand{\ibold}{{\bf i}}
\newcommand{\jbold}{{\bf j}}
\newcommand{\ebold}{{\bf e}}
\newcommand{\vbold}{{\bf v}}
\newcommand{\xbold}{{\bf x}}
\newcommand{\ubold}{{\bf u}}
\newcommand{\nbold}{{\bf n}}
\newcommand{\Fbold}{{\bf F}}
\newcommand{\ubar}{{\bar {\bf u}}}
\newcommand{\vb}{{\bf u}}
\newcommand{\npo}{{n + 1}}
\newcommand{\dt}{{\Delta t}}
\newcommand{\dx}{{h}}
\newcommand{\nph}{{n + \frac{1}{2}}}
\newcommand{\half}{\frac{1}{2}}
\newcommand{\iph}{{\ibold + \half \ebold^d}}
\newcommand{\imh}{{\ibold - \half \ebold^d}}


\title{Scalar Advection-Diffusion Example in AMReX}

\author{Dan Graves}
\begin{document}

\begin{abstract}
  The advection-diffusion example is intended to explore how one does
  a higher order algorithm in AMReX.    The example consists of an
  $\tt{AmrLevel}$-derived class with several variations of
  advection-diffusion algorithms implemented and tested.  These
  include a second-order Godunov algorithm, as well second, third, and
  fourth order Runge-Kutta.   Diffusion terms are updated explicitly,
  making this example only appropriate for sufficiently small
  diffusion coefficients.
\end{abstract}

\section{Introduction}

Given a velocity field $\vbold$, a scalar field  $\phi$ and a
diffusion coefficient $\nu$, the standard way to write the 
scalar advection-diffusion equation  is given by 
$$
\frac{\partial \phi}{\partial t} + \nabla \cdot (\vbold \phi) = \nabla
(\nu \nabla \phi).
\cot 
$$
For this application, we are assuming that 
\begin{itemize}
\item The diffusion coefficient is constant.  This would be fairly
  trivial to change.
\item For any grid spacing $\dx$ we intend to us, the diffusion
  coefficient $\nu$ is sufficiently small that the time
  step $\dt$ is controlled by the advective CFL condition
$$
\dt < \frac{\dx}{v_m}.
$$
where $v_m= \max|\vbold|$;
Roughly speaking, this means that 
$$
\nu < v_m \dx.
$$
\end{itemize}
Given that we will updating the diffusion explicitly, we can
reformulate the equation as 
$$
\frac{\partial \phi}{\partial t} =  -\nabla \cdot \fbold, 
$$
where $\fbold = \vbold \phi - \nu \nabla phi$.
The algorithms here are implemented in finite volume form:
$$
D(F)_\ibold = \frac{1}{dx} \sum^D_{d = 1} (F_{\iph} - F_\imh)
$$
and are
therefore strongly conservative.

\section{Algorithm Choices}

\subsection{Second Order Godunov} 

In second-order Godunov, we extrapolate in space and time to get a
flux at the half time step.   I will not go into great detail here
because this is fairly old hat to my audience.
$$
\phi_{i+1/2} = \phi_i + \half\dx \partial{\phi}{\partial x} + \half\dt \partial{\phi}{\partial t}
$$
We use the equation of motion to substitute spatial derivatives for
the time derivative and we get.   We use  limited van Leer slopes to
approximate the spatial derivatives.  We extrapolate to every face
from both sides and pick the upwind state.   For the diffusive flux,
we simply take a centered difference of $\phi^n$ at the face.  So, for
diffusion, in this algorithm, we are first order in time.   Since we
are only working with small diffusion coefficients, it does not seem
to matter to the convergence numbers.

\subsection{RK2}
\subsection{RK3 (TVD)}
\subsection{RK4}

\bibliography{references}

\end{document}

