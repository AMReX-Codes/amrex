% Copyright 2017-2019 Jean-Luc Vay, Remi Lehe
%
% This file is part of WarpX.
%
% License: BSD-3-Clause-LBNL


\usepackage{bm}
\usepackage{amsmath}
\usepackage{amssymb}
\usepackage{graphicx}
\usepackage{url}
\usepackage{hyperref}

\usepackage[displaymath]{lineno}\usepackage{bm}% bold math

\newcommand{\fe}{\mathbf{\tilde{E}}}
\newcommand{\fb}{\mathbf{\tilde{B}}}
\newcommand{\fj}{\mathbf{\tilde{J}}}
\newcommand{\ff}{\tilde{F}}
\newcommand{\fg}{\tilde{G}}
\newcommand{\fk}{\mathbf{k}}
\newcommand{\fkhat}{\mathbf{\hat{k}}}

% Definitions from Remi's paper on Galilean math
\newcommand{\Km}{\vec{K}_{\vec{m}}}
\newcommand{\km}{\vec{k}_{\vec{m}}}
\renewcommand{\vec}[1]{\boldsymbol{#1}}
\newcommand{\vgal}{\vec{v}_{gal}}
\newcommand{\nab}{\vec{\nabla'}}
\newcommand{\Dt}[1]{ \frac{\partial #1}{\partial t}}
\newcommand{\mc}[1]{\hat{\mathcal{#1}}}
\newcommand{\xj}{\vec{x}'_{\vec{j}}}
\newcommand{\Xll}{\vec{X}_{\vec{\ell}}}
\newcommand{\Integ}[1]{\int_{-\infty}^{\infty} \!\!\!\!\!\!
  \mathrm{d}#1}
\newcommand{\RInteg}[1]{\int_{0}^{\infty} \!\! \frac{#1\mathrm{d}#1}{(2\pi)^2}}

% Definitions from Remi's Thesis
\newcommand{\h}{\mathcal{H}}
\newcommand{\hf}{\frac{1}{2}}
\newcommand{\um}{$\mu$m}
\newcommand{\Um}{\mu \mathrm{m}}
\newcommand{\aal}{\langle \vec{a}_l^2 \rangle}
\newcommand{\etad}{ \eta_d }
\newcommand{\etae}{ \eta_\epsilon }
\newcommand{\etag}{ \eta_\gamma }
\newcommand{\tlambda}{ \tilde{\lambda} }
%\newcommand\comment[1]{\textcolor{red}{\textbf{#1}}}
\newcommand{\gsim}{\mathrel{\hbox{\rlap{\lower.55ex
\hbox{$\sim$}} \kern-.3em \raise.4ex \hbox{$>$}}}}
\newcommand{\lsim}{\mathrel{\hbox{\rlap{\lower.55ex
\hbox{$\sim$}} \kern-.3em \raise.4ex \hbox{$<$}}}}
\newcommand{\kfoc}{k_\mathrm{foc}}
\newcommand{\bkfoc}{\bar{k}_\mathrm{foc}}
\newcommand{\xil}{\xi_{\mathrm{laser}}}

\newcommand{\Ex}[2]{{E_x}^{#1}_{#2}}
\newcommand{\Ey}[2]{{E_y}^{#1}_{#2}}
\newcommand{\Ez}[2]{{E_z}^{#1}_{#2}}
\newcommand{\Bx}[2]{{B_x}^{#1}_{#2}}
\newcommand{\By}[2]{{B_y}^{#1}_{#2}}
\newcommand{\Bz}[2]{{B_z}^{#1}_{#2}}
\newcommand{\Jx}[2]{{J_x}^{#1}_{#2}}
\newcommand{\Jy}[2]{{J_y}^{#1}_{#2}}
\newcommand{\Jz}[2]{{J_z}^{#1}_{#2}}

\newcommand{\tEr}[2]{\tilde{E_r}^{#1}_{#2}}
\newcommand{\tEt}[2]{\tilde{E_\theta}^{#1}_{#2}}
\newcommand{\tEz}[2]{\tilde{E_z}^{#1}_{#2}}
\newcommand{\tBr}[2]{\tilde{B_r}^{#1}_{#2}}
\newcommand{\tBt}[2]{\tilde{B_\theta}^{#1}_{#2}}
\newcommand{\tBz}[2]{\tilde{B_z}^{#1}_{#2}}
\newcommand{\tJr}[2]{\tilde{J_r}^{#1}_{#2}}
\newcommand{\tJt}[2]{\tilde{J_\theta}^{#1}_{#2}}
\newcommand{\tJz}[2]{\tilde{J_z}^{#1}_{#2}}

\newcommand{\CCirc}{\textsc{Calder Circ}}
\newcommand{\CCart}{\textsc{Calder 3D}}

It was shown in \cite{Vaypop2008} that the Boris formulation is
not Lorentz invariant and can lead to significant errors in the treatment
of relativistic dynamics. A Lorentz invariant formulation is obtained
by considering the following velocity average 
\begin{align}
\mathbf{\bar{v}}^{i}= & \frac{\mathbf{v}^{i+1/2}+\mathbf{v}^{i-1/2}}{2},\label{Eq:new_v}
\end{align}
This gives a system that is solvable analytically (see \cite{Vaypop2008}
for a detailed derivation), giving the following velocity update:

\begin{subequations}
\begin{align}
\mathbf{u^{*}}= & \mathbf{u}^{i-1/2}+\frac{q\Delta t}{m}\left(\mathbf{E}^{i}+\frac{\mathbf{v}^{i-1/2}}{2}\times\mathbf{B}^{i}\right),\label{pusher_gamma}\\
\mathbf{u}^{i+1/2}= & \left[\mathbf{u^{*}}+\left(\mathbf{u^{*}}\cdot\mathbf{t}\right)\mathbf{t}+\mathbf{u^{*}}\times\mathbf{t}\right]/\left(1+t^{2}\right),\label{pusher_upr}
\end{align}
\end{subequations}
where $\mathbf{t}=\bm{\tau}/\gamma^{i+1/2}$, $\bm{\tau}=\left(q\Delta t/2m\right)\mathbf{B}^{i}$,
$\gamma^{i+1/2}=\sqrt{\sigma+\sqrt{\sigma^{2}+\left(\tau^{2}+w^{2}\right)}}$,
$w=\mathbf{u^{*}}\cdot\bm{\tau}$, $\sigma=\left(\gamma'^{2}-\tau^{2}\right)/2$
and $\gamma'=\sqrt{1+(\mathbf{u}^{*}/c)^{2}}$. This Lorentz invariant formulation
is particularly well suited for the modeling of ultra-relativistic
charged particle beams, where the accurate account of the cancellation
of the self-generated electric and magnetic fields is essential, as
shown in \cite{Vaypop2008}.
