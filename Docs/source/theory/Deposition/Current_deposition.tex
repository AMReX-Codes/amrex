% Copyright 2017-2019 Jean-Luc Vay, Remi Lehe
%
% This file is part of WarpX.
%
% License: BSD-3-Clause-LBNL


\usepackage{bm}
\usepackage{amsmath}
\usepackage{amssymb}
\usepackage{graphicx}
\usepackage{url}
\usepackage{hyperref}

\usepackage[displaymath]{lineno}\usepackage{bm}% bold math

\newcommand{\fe}{\mathbf{\tilde{E}}}
\newcommand{\fb}{\mathbf{\tilde{B}}}
\newcommand{\fj}{\mathbf{\tilde{J}}}
\newcommand{\ff}{\tilde{F}}
\newcommand{\fg}{\tilde{G}}
\newcommand{\fk}{\mathbf{k}}
\newcommand{\fkhat}{\mathbf{\hat{k}}}

% Definitions from Remi's paper on Galilean math
\newcommand{\Km}{\vec{K}_{\vec{m}}}
\newcommand{\km}{\vec{k}_{\vec{m}}}
\renewcommand{\vec}[1]{\boldsymbol{#1}}
\newcommand{\vgal}{\vec{v}_{gal}}
\newcommand{\nab}{\vec{\nabla'}}
\newcommand{\Dt}[1]{ \frac{\partial #1}{\partial t}}
\newcommand{\mc}[1]{\hat{\mathcal{#1}}}
\newcommand{\xj}{\vec{x}'_{\vec{j}}}
\newcommand{\Xll}{\vec{X}_{\vec{\ell}}}
\newcommand{\Integ}[1]{\int_{-\infty}^{\infty} \!\!\!\!\!\!
  \mathrm{d}#1}
\newcommand{\RInteg}[1]{\int_{0}^{\infty} \!\! \frac{#1\mathrm{d}#1}{(2\pi)^2}}

% Definitions from Remi's Thesis
\newcommand{\h}{\mathcal{H}}
\newcommand{\hf}{\frac{1}{2}}
\newcommand{\um}{$\mu$m}
\newcommand{\Um}{\mu \mathrm{m}}
\newcommand{\aal}{\langle \vec{a}_l^2 \rangle}
\newcommand{\etad}{ \eta_d }
\newcommand{\etae}{ \eta_\epsilon }
\newcommand{\etag}{ \eta_\gamma }
\newcommand{\tlambda}{ \tilde{\lambda} }
%\newcommand\comment[1]{\textcolor{red}{\textbf{#1}}}
\newcommand{\gsim}{\mathrel{\hbox{\rlap{\lower.55ex
\hbox{$\sim$}} \kern-.3em \raise.4ex \hbox{$>$}}}}
\newcommand{\lsim}{\mathrel{\hbox{\rlap{\lower.55ex
\hbox{$\sim$}} \kern-.3em \raise.4ex \hbox{$<$}}}}
\newcommand{\kfoc}{k_\mathrm{foc}}
\newcommand{\bkfoc}{\bar{k}_\mathrm{foc}}
\newcommand{\xil}{\xi_{\mathrm{laser}}}

\newcommand{\Ex}[2]{{E_x}^{#1}_{#2}}
\newcommand{\Ey}[2]{{E_y}^{#1}_{#2}}
\newcommand{\Ez}[2]{{E_z}^{#1}_{#2}}
\newcommand{\Bx}[2]{{B_x}^{#1}_{#2}}
\newcommand{\By}[2]{{B_y}^{#1}_{#2}}
\newcommand{\Bz}[2]{{B_z}^{#1}_{#2}}
\newcommand{\Jx}[2]{{J_x}^{#1}_{#2}}
\newcommand{\Jy}[2]{{J_y}^{#1}_{#2}}
\newcommand{\Jz}[2]{{J_z}^{#1}_{#2}}

\newcommand{\tEr}[2]{\tilde{E_r}^{#1}_{#2}}
\newcommand{\tEt}[2]{\tilde{E_\theta}^{#1}_{#2}}
\newcommand{\tEz}[2]{\tilde{E_z}^{#1}_{#2}}
\newcommand{\tBr}[2]{\tilde{B_r}^{#1}_{#2}}
\newcommand{\tBt}[2]{\tilde{B_\theta}^{#1}_{#2}}
\newcommand{\tBz}[2]{\tilde{B_z}^{#1}_{#2}}
\newcommand{\tJr}[2]{\tilde{J_r}^{#1}_{#2}}
\newcommand{\tJt}[2]{\tilde{J_\theta}^{#1}_{#2}}
\newcommand{\tJz}[2]{\tilde{J_z}^{#1}_{#2}}

\newcommand{\CCirc}{\textsc{Calder Circ}}
\newcommand{\CCart}{\textsc{Calder 3D}}

The current densities are deposited on the computational grid from
the particle position and velocities, employing splines of various
orders \cite{Abejcp86}.

\begin{subequations}
\begin{eqnarray}
\rho & = & \frac{1}{\Delta x \Delta y \Delta z}\sum_nq_nS_n\\
\mathbf{J} & = & \frac{1}{\Delta x \Delta y \Delta z}\sum_nq_n\mathbf{v_n}S_n
\end{eqnarray}
\end{subequations}

In most applications, it is essential to prevent the accumulation
of errors resulting from the violation of the discretized Gauss' Law.
This is accomplished by providing a method for depositing the current
from the particles to the grid that preserves the discretized Gauss'
Law, or by providing a mechanism for ``divergence cleaning'' \cite{Birdsalllangdon,Langdoncpc92,Marderjcp87,Vaypop98,Munzjcp2000}.
For the former, schemes that allow a deposition of the current that
is exact when combined with the Yee solver is given in \cite{Villasenorcpc92}
for linear splines and in \cite{Esirkepovcpc01} for splines of arbitrary order. 

The NSFDTD formulations given above and in \cite{PukhovJPP99,Vayjcp2011,CowanPRSTAB13,LehePRSTAB13} 
apply to the Maxwell-Faraday
equation, while the discretized Maxwell-Ampere equation uses the FDTD
formulation. Consequently, the charge conserving algorithms developed
for current deposition \cite{Villasenorcpc92,Esirkepovcpc01} apply
readily to those NSFDTD-based formulations. More details concerning
those implementations, including the expressions for the numerical
dispersion and Courant condition are given 
in \cite{PukhovJPP99,Vayjcp2011,CowanPRSTAB13,LehePRSTAB13}. 

In the case of the pseudospectral solvers, the current deposition
algorithm generally does not satisfy the discretized continuity equation
in Fourier space $\tilde{\rho}^{n+1}=\tilde{\rho}^{n}-i\Delta t\fk\cdot\mathbf{\tilde{J}}^{n+1/2}$.
In this case, a Boris correction \cite{Birdsalllangdon} can be applied
in $k$ space in the form $\fe_{c}^{n+1}=\fe^{n+1}-\left(\fk\cdot\fe^{n+1}+i\tilde{\rho}^{n+1}\right)\fkhat/k$,
where $\fe_{c}$ is the corrected field. Alternatively, a correction
to the current can be applied (with some similarity to the current
deposition presented by Morse and Nielson in their potential-based
model in \cite{Morsenielson1971}) using $\fj_{c}^{n+1/2}=\fj^{n+1/2}-\left[\fk\cdot\fj^{n+1/2}-i\left(\tilde{\rho}^{n+1}-\tilde{\rho}^{n}\right)/\Delta t\right]\fkhat/k$,
where $\fj_{c}$ is the corrected current. In this case, the transverse
component of the current is left untouched while the longitudinal
component is effectively replaced by the one obtained from integration
of the continuity equation, ensuring that the corrected current satisfies
the continuity equation. The advantage of correcting the current rather than 
the electric field is that it is more local and thus more compatible with 
domain decomposition of the fields for parallel computation \cite{VayJCP2013}.

Alternatively, an exact current deposition can be written for the pseudospectral solvers, following the geometrical interpretation of existing methods in real space \cite{Morsenielson1971,Villasenorcpc92,Esirkepovcpc01}, thereby averaging the currents of the paths following grid lines between positions $(x^n,y^n)$ and $(x^{n+1},y^{n+1})$, which is given in 2D (extension to 3D follows readily) for $k\neq0$ by  \cite{VayJCP2013}:
%
\begin{eqnarray}
\fj^{k\neq0}=\frac{i\mathbf{\tilde{D}}}{\fk}
\label{Eq_Jdep_1}
\end{eqnarray}
with 
\begin{eqnarray}
D_x   =  \frac{1}{2\Delta t}\sum_i q_i
  [\Gamma(x_i^{n+1},y_i^{n+1})-\Gamma(x_i^{n},y_i^{n+1}) \nonumber\\ 
+\Gamma(x_i^{n+1},y_i^{n})-\Gamma(x_i^{n},y_i^{n})],\\
D_y   =  \frac{1}{2\Delta t}\sum_i q_i
  [\Gamma(x_i^{n+1},y_i^{n+1})-\Gamma(x_i^{n+1},y_i^{n}) \nonumber \\
+\Gamma(x_i^{n},y_i^{n+1})-\Gamma(x_i^{n},y_i^{n})],
\end{eqnarray}
where $\Gamma$ is the macro-particle form factor. 
%
The contributions for $k=0$ are integrated directly in real space  \cite{VayJCP2013}.
