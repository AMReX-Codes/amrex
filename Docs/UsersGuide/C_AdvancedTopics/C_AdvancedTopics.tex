Here we will discuss some features of the C++ in greater detail. In 
particular we will examine \BoxLib's passively advected particles.

\section{Particles}\label{Sec:Particles}
[Note from Ethan: This section is currently being expanded. Expect information to be 
incomplete and occasionally inaccurate]

\subsection{ParticleBase and Particle}
{\tt ParticleBase} provides the data {\tt struct} upon which all 
particles are built. Each particle has integer values {\tt m\_id}, {\tt 
m\_cpu}, {\tt m\_lev}, and {\tt m\_grid} that store the particles id, 
host cpu, grid level and grid id respectively. It also has an {\tt 
IntVect} that stores its cell position and a $d$ length real vector 
storing its spatial position.

{\tt ParticleBase} also provides basic methods and methods for interfacing between 
the particle and the grids. These include methods that check the 
particle position, transfer the particle between grids, and 
interpolate data appropriately.

{\tt Particle<N>} is a template class that adds a (real) metadata array of 
user-specified size to the {\tt ParticleBase}. For instance, this 
metadata might be 
\begin{enumerate}
   \item particle mass
   \item particle x velocity
   \item particle y velocity
   \item particle z velocity
\end{enumerate}
\subsection{ParticleContainer}
{\tt ParticleContainer} is a template class that stores a number of 
{\tt Particle<N>}s. These particles are stored on a level by level and 
grid by grid basis to make accessing and performing operations on them 
more memory efficient. The class also provided methods that move and 
redistribute the stored particles.
\section{State Data}
