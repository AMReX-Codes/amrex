
In this chapter, we discuss \amrex's build systems.  There are three
ways to use \amrex.  The approach used by \amrex\ developers uses GNU
Make.  There is no installation step in this approach.  Application
codes adopt \amrex's build system and compile \amrex\ while compiling
their own codes.  This will be discussed in more details in
Section~\ref{sec:build:make}.  The second approach is to build \amrex\
into a library and install it (Section~\ref{sec:build:lib}).  Then an
application code uses its own build system and links \amrex\ as an
external library.  \amrex\ can also be built with CMake
(Section~\ref{sec:build:cmake}).

\section{Building with GNU Make}
\label{sec:build:make}

In this build approach, you write your own make files defining a
number of variables and rules.  Then you invoke {\tt make} to start
the building process.  This will result in an executable upon
successful completion.  The temporary files generated in the building
process are stored in a temporary directory named {\tt
  tmp\_build\_dir}.

\subsection{Dissecting a Simple Make File}

An example of building with GNU Make can be found in {\tt
  amrex/Tutorials/Basic/HelloWorld\_C}.  Table~\ref{tab:makevarimp}
shows a list of important variables.
\begin{table}[t]
  \centering
  \begin{tabular}{lcc}
    Variable & Value & Default \\
    \hline
    {\tt AMREX\_HOME} & Path to amrex & environment \\
    {\tt COMP} & {\tt gnu}, {\tt cray}, {\tt intel}, {\tt llvm}, or {\tt pgi} & none \\
    {\tt DEBUG} & {\tt TRUE} or {\tt FALSE} & {\tt TRUE} \\
    {\tt DIM} & 1 or 2 or 3 & none \\
    {\tt USE\_MPI} & {\tt TRUE} or {\tt FALSE} & {\tt FALSE} \\
    {\tt USE\_OMP} & {\tt TRUE} or {\tt FALSE} & {\tt FALSE} \\
    \hline
  \end{tabular}
  \caption{\label{tab:makevarimp} Important make variables}
\end{table}

At the beginning of {\tt
  amrex/Tutorials/Basic/HelloWorld\_C/GNUmakefile}, {\tt AMREX\_HOME}
is set to the path to the top directory of \amrex.  Note that in the
example {\tt ?=} is a conditional variable assignment operator that
only has an effect if {\tt AMREX\_HOME} has not been defined
(including in the environment).  One can also set it through
environment variable.  For example in bash, one can set {\tt export
  AMREX\_HOME=/path/to/amrex}.

One must set the {\tt COMP} variable to choose a compiler.  Currently
the list of supported compilers includes {\tt gnu}, {\tt cray}, {\tt
  intel}, {\tt llvm}, and {\tt pgi}.  One must also set the {\tt DIM}
variable to either 1, 2, or 3, depending on the dimensionality of the
problem. 

Variables {\tt DEBUG}, {\tt USE\_MPI} and {\tt USE\_OMP} are optional
with default set to {\tt TRUE}, {\tt FALSE} and {\tt FALSE},
respectively.  The meaning of these variables should be obvious.  In
{\tt DEBUG} build, aggressive compiler optimization flags are turned
off and assertions in \amrex\ source code are turned on.  For
production runs, {\tt DEBUG} should be set to {\tt FALSE}.

After defining these make variables, a number of files, {\tt
  Make.defs}, {\tt Make.package} and {\tt Make.rules}, are included in
{\tt GNUmakefile}.  \amrex\ are split into a number of packages.  An
application code does not include every package.  In this simple
example, we only need to include {\tt
  \$(AMREX\_HOME)/Src/Base/Make.package}.  An application code also
has its own {\tt Make.package} file (e.g., {\tt ./Make.package} in
this example) to append source files to the build system using
operator {\tt +=}.  Variables for various source files are shown
below.
\begin{quote}
\begin{description}
\item [{CEXE\_sources}] \cpp\ source files.  Note that \cpp\
  source files are assumed to have a {\tt .cpp} extension.
\item [{CEXE\_headers}] \cpp\ headers with {\tt .h} or {\tt .H} extension.
\item [{cEXE\_sources}] C source files with {\tt .c} extension.
\item [{cEXE\_headers}] C headers with {\tt .h} extension.
\item [{f90EXE\_sources}] Free format Fortran source with {\tt
    .f90} extension.
\item [{F90EXE\_sources}] Free format Fortran source with {\tt
    .F90} extension.  Note that these Fortran files will go through
  preprocessing.
\end{description}
\end{quote}
In this simple example, the extra source file, {\tt main.cpp} is in
the current directory that is already in the build system's search
path.  If this example has files in a subdirectory (e.g., {\tt
  mysrcdir}), you will then need to add the following to {\tt
  Make.package}. 
\begin{verbatim}
    VPATH_LOCATIONS += mysrcdir
    INCLUDE_LOCATIONS += mysrcdir
\end{verbatim}
Here {\tt VPATH\_LOCATIONS} and {\tt INCLUDE\_LOCATIONS} are the search
path for source and header files, respectively.

\subsection{Tweaking Make System}

The GNU Make build system is located at {\tt amrex/Tools/GNUMake}.
You can read {\tt README.md} and the make files there for more
information.  Here we will give a brief overview.

Besides building executable, other common make commands include:
\begin{quote}
\begin{description}
  \item[make clean] This removes the executable, {\tt .o} files, and
    the temporarily generated files.  Note that one can add additional
    targets to this rule by using the double colon (::)
  \item[make realclean] This removes all files generated by make.
  \item[make help] This shows the rules for compilation.
  \item[make print-xxx] shows the value of varible {\tt xxx}.  This is
    very useful for debugging and tweaking the make system.
\end{description}
\end{quote}

Compiler flags are set in {\tt amrex/Tools/GNUMake/comps/}.  Note that
variables like {\tt CC} and {\tt CFLAGS} are reset in that directory
and their values in environment variables are disregarded. Site
specific setups (e.g., MPI installation) are in {\tt
  amrex/Tools/GNUMake/sites/}, which includes a generic setup in {\tt
  Make.unknown}.  You can override the setup by having your own {\tt
  Make.\$(host\_name)} file, where variable {\tt host\_name} is your
host name in the make system and can be found via {\tt make
  print-host\_name}.  You can also have a {\tt
  amrex/Tools/GNUMake/Make.local} file to override various variables.
See {\tt amrex/Tools/GNUMake/Make.local.template} for an example.

\section{Building {\tt libamrex}}
\label{sec:build:lib}

If an application code already has its own elaborated build system and
wants to use \amrex\ as an external library, this might be your
choice.  In this approach, one runs {\tt ./configure}, followed by
{\tt make} and {\tt make install}.  In the top \amrex\ directory, one
can run {\tt ./configure -h} to show the various options for the {\tt
  configure} script.  This approach is built on the \amrex\ GNU Make
system.  Thus Section~\ref{sec:build:make} is recommended if any fine
tuning is needed.

\section{Building with CMake}
\label{sec:build:cmake}

\MarginPar{todo}

