In this chapter, we will walk you through two simple examples.  It is
assumed here that your machine has GNU Make, Python, GCC (including
gfortran), and MPI, although \amrex\ can be built with CMake and other
compilers. 

\section{Downloading the Code}

The source code of \amrex\ is available at
\url{https://github.com/AMReX-Codes/amrex}.  The GitHub repo is our
central repo for development.  The {\tt development} branch
includes the latest state of the code, and it is merged into the {\tt
  master} branch on a monthly basis.  The {\tt master} branch is
considered the release branch.  The releases are tagged with version
number {\tt YY.MM} (e.g., {\tt 17.04}).  The {\tt MM} part of the
version is incremented every month, and the {\tt YY} part every year.
Bug fix releases are tagged with {\tt YY.MM.patch} (e.g., {\tt
  17.04.1}).

\section{Example: Hello World}

The source code of this example is at {\tt
  amrex/Tutorials/HelloWorld\_C/} and is also shown below. 

\begin{lstlisting}[language=cpp]
#include <AMReX.H>
#include <AMReX_Print.H>

int main(int argc, char* argv[])
{
    amrex::Initialize(argc,argv);
    amrex::Print() << "Hello world from AMReX version " 
                   << amrex::Version() << "\n";
    amrex::Finalize();
}
\end{lstlisting}

The main body of this short example contains three statements.
Usually the first and last statements for the {\tt main} function of
every program should be calling {\tt amrex::}\idxamrex{Initialize} and
\idxamrex{Finalize}, respectively.  The second statement calls {\tt
  amrex::}\idxamrex{Print} to print out a string that includes the
\amrex\ version returned by the {\tt amrex::}\idxamrex{Version}
function.  The example code includes two \amrex\ header files.  Note
that the name of all \amrex\ header files starts with {\tt AMReX\_}
(or just {\tt AMReX} in the case of {\tt AMReX.H}).  All \amrex\
\cpp\ functions are in the {\tt amrex} namespace.  

\subsection{Building the Code}

You build the code in the {\tt amrex/Tutorials/HelloWorld\_C/}
directory.  Typing {\tt make} will start the compilation process and
result in an executable named {\tt main3d.gnu.DEBUG.ex}.  The name
shows that the GNU compiler with debug options set by \amrex\ is used.
It also shows that the executable is built for 3D.  Although this
simple example code is dimension independent, the dimension matters
for all non-trivial examples.  The build process can be adjusted by
modifying the {\tt amrex/Tutorials/HellWorld\_C/GNUmakefile} file.
More details on how to build \amrex can be found in
Chapter~\ref{Chap:Building AMReX}.

\subsection{Running the Code}

The example code can be run as follows,
\begin{verbatim}
  ./main3d.gnu.DEBUG.ex
\end{verbatim}
The result looks like,
\begin{verbatim}
  Hello world from AMReX version 17.05-30-g5775aed933c4-dirty
\end{verbatim}



\subsection{Exercise}

MPI  note that mpi is initialize inside initialize

paralleldescriptor

allprint

\section{Example: Heat Equation Solver}

\subsection{Building the Code}

how to compile

\subsection{Running the Code}

how to run

\subsection{Visualization of the Results}

how to visualize plotfiles

